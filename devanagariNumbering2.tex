%%%%%%%%%%%%%%%
% Define Devanāgarī numerals. This will translate Arabic (Roman) numbers to Devanāgarī. 
\makeatletter
\newcommand{\devanagarinumeral}[1]{\expandafter\@devanagarinumeral\csname c@#1\endcsname}
\newcommand{\@devanagarinumeral}[1]{%
  \ifcase#1\or
  १\or २\or ३\or ४\or ५\or ६\or ७\or ८\or ९\or १०\or
  ११\or १२\or १३\or १४\or १५\or १६\or १७\or १८\or १९\or २०\or
  २१\or २२\or २३\or २४\or २५\or २६\or २७\or २८\or २९\or ३०\or
  ३१\or ३२\or ३३\or ३४\or ३५\or ३६\or ३७\or ३८\or ३९\or ४०\or
  ४१\or ४२\or ४३\or ४४\or ४५\or ४६\or ४७\or ४८\or ४९\or ५०\or
  ५१\or ५२\or ५३\or ५४\or ५५\or ५६\or ५७\or ५८\or ५९\or ६०\or
  ६१\or ६२\or ६३\or ६४\or ६५\or ६६\or ६७\or ६८\or ६९\or ७०\or
  ७१\or ७२\or ७३\or ७४\or ७५\or ७६\or ७७\or ७८\or ७९\or ८०\or
  ८१\or ८२\or ८३\or ८४\or ८५\or ८६\or ८७\or ८८\or ८९\or ९०\or
  ९१\or ९२\or ९३\or ९४\or ९५\or ९६\or ९७\or ९८\or ९९\or १००\else
  \@ctrerr\fi
}

% Define Devanāgarī decimal numerals
\newcommand{\devanagaridecimal}[1]{\expandafter\@devanagaridecimal#1\@nil}
\def\@devanagaridecimal#1.#2\@nil{%
  \devanagarinumeral{#1}.\devanagarinumeral{#2}%
}
\renewcommand{\thesection}{\devanagarinumeral{section}}
\renewcommand{\thesubsection}{\devanagarinumeral{section}.\devanagarinumeral{subsection}}
\renewcommand{\thesubsubsection}{\devanagarinumeral{section}.\devanagarinumeral{subsection}.\devanagarinumeral{subsubsection}}
\makeatother

% Mapping for Devanāgarī page numbers. Using the \newcommand macro to define page numbers manually. 
\newcommand{\devanagaripagenumber}[1]{%
  \ifcase#1\or
  १\or २\or ३\or ४\or ५\or ६\or ७\or ८\or ९\or १०\or
  ११\or १२\or १३\or १४\or १५\or १६\or १७\or १८\or १९\or २०\or
  २१\or २२\or २३\or २४\or २५\or २६\or २७\or २८\or २९\or ३०\or
  ३१\or ३२\or ३३\or ३४\or ३५\or ३६\or ३७\or ३८\or ३९\or ४०\or
  ४१\or ४२\or ४३\or ४४\or ४५\or ४६\or ४७\or ४८\or ४९\or ५०\or
  ५१\or ५२\or ५३\or ५४\or ५५\or ५६\or ५७\or ५८\or ५९\or ६०\or
  ६१\or ६२\or ६३\or ६४\or ६५\or ६६\or ६७\or ६८\or ६९\or ७०\or
  ७१\or ७२\or ७३\or ७४\or ७५\or ७६\or ७७\or ७८\or ७९\or ८०\or
  ८१\or ८२\or ८३\or ८४\or ८५\or ८६\or ८७\or ८८\or ८९\or ९०\or
  ९१\or ९२\or ९३\or ९४\or ९५\or ९६\or ९७\or ९८\or ९९\or १००\else
  #1\fi
}

%%%%%% For Enumerating in Devanāgarī numerals
\makeatletter
\newcommand{\devanagariEnumeral}[1]{\expandafter\@devanagariEnumeral\csname c@#1\endcsname}
\newcommand{\@devanagariEnumeral}[1]{%
  \ifcase#1\or
  १\or २\or ३\or ४\or ५\or ६\or ७\or ८\or ९\or १०\or
  ११\or १२\or १३\or १४\or १५\or १६\or १७\or १८\or १९\or २०\or
  २१\or २२\or २३\or २४\or २५\or २६\or २७\or २८\or २९\or ३०\or
  ३१\or ३२\or ३३\or ३४\or ३५\or ३६\or ३७\or ३८\or ३९\or ४०\or
  ४१\or ४२\or ४३\or ४४\or ४५\or ४६\or ४७\or ४८\or ४९\or ५०\or
  ५१\or ५२\or ५३\or ५४\or ५५\or ५६\or ५७\or ५८\or ५९\or ६०\or
  ६१\or ६२\or ६३\or ६४\or ६५\or ६६\or ६७\or ६८\or ६९\or ७०\or
  ७१\or ७२\or ७३\or ७४\or ७५\or ७६\or ७७\or ७८\or ७९\or ८०\or
  ८१\or ८२\or ८३\or ८४\or ८५\or ८६\or ८७\or ८८\or ८९\or ९०\or
  ९१\or ९२\or ९३\or ९४\or ९५\or ९६\or ९७\or ९८\or ९९\or १००\else
  \@ctrerr\fi
}
\makeatother

\AddEnumerateCounter{\devanagariEnumeral}{\@devanagariEnumeral}{०}

\setlist[enumerate]{label=\devanagarifont\devanagariEnumeral*., before=\devanagarifont, leftmargin=2.5em}

%% For Equation Numbering
\usepackage{etoolbox}

\makeatletter
\newcommand{\devanagariEquation}[1]{\expandafter\@devanagariEquation\csname c@#1\endcsname}
\newcommand{\@devanagariEquation}[1]{%
  \ifcase#1\or
  १\or २\or ३\or ४\or ५\or ६\or ७\or ८\or ९\or १०\or
  ११\or १२\or १३\or १४\or १५\or १६\or १७\or १८\or १९\or २०\or
  २१\or २२\or २३\or २४\or २५\or २६\or २७\or २८\or २९\or ३०\or
  ३१\or ३२\or ३३\or ३४\or ३५\or ३६\or ३७\or ३८\or ३९\or ४०\or
  ४१\or ४२\or ४३\or ४४\or ४५\or ४६\or ४७\or ४८\or ४९\or ५०\or
  ५१\or ५२\or ५३\or ५४\or ५५\or ५६\or ५७\or ५८\or ५९\or ६०\or
  ६१\or ६२\or ६३\or ६४\or ६५\or ६६\or ६७\or ६८\or ६९\or ७०\or
  ७१\or ७२\or ७३\or ७४\or ७५\or ७६\or ७७\or ७८\or ७९\or ८०\or
  ८१\or ८२\or ८३\or ८४\or ८५\or ८६\or ८७\or ८८\or ८९\or ९०\or
  ९१\or ९२\or ९३\or ९४\or ९५\or ९६\or ९७\or ९८\or ९९\or १००\else
  \@ctrerr\fi
}

\renewcommand\theequation{\devanagariEquation{equation}}
\makeatother



% Helper function to convert Arabic to Devanāgarī numerals in TOC
\newcommand{\convertpage}[1]{%
  \ifcase#1\or
  १\or २\or ३\or ४\or ५\or ६\or ७\or ८\or ९\or १०\or
  ११\or १२\or १३\or १४\or १५\or १६\or १७\or १८\or १९\or २०\or
  २१\or २२\or २३\or २४\or २५\or २६\or २७\or २८\or २९\or ३०\or
  ३१\or ३२\or ३३\or ३४\or ३५\or ३६\or ३७\or ३८\or ३९\or ४०\or
  ४१\or ४२\or ४३\or ४४\or ४५\or ४६\or ४७\or ४८\or ४९\or ५०\or
  ५१\or ५२\or ५३\or ५४\or ५५\or ५६\or ५७\or ५८\or ५९\or ६०\or
  ६१\or ६२\or ६३\or ६४\or ६५\or ६६\or ६७\or ६८\or ६९\or ७०\or
  ७१\or ७२\or ७३\or ७४\or ७५\or ७६\or ७७\or ७८\or ७९\or ८०\or
  ८१\or ८२\or ८३\or ८४\or ८५\or ८६\or ८७\or ८८\or ८९\or ९०\or
  ९१\or ९२\or ९३\or ९४\or ९५\or ९६\or ९७\or ९८\or ९९\or १००\else
}


%%%%%%%%%%%%%  Table of contents in Devanāgarī %%%%%%%%%%%%%%
% Redefine the \@dottedtocline to convert page numbers of subsection to Devanāgarī
\makeatletter
\def\@dottedtocline#1#2#3#4#5{%
  \ifnum #1>\c@tocdepth \else
    \vskip \z@ \@plus.2\p@
    {\leftskip #2\relax \rightskip #3\relax
      \parfillskip \@pnumwidth \@tempdima #3\relax
      \leavevmode
      \@tempdima 1cm\relax % adjust identation of subsection title
      \advance\leftskip \@tempdima \null\hskip -\leftskip
      {#4}\nobreak\leaders\hbox{$\m@th
        \mkern \@dotsep mu\hbox{.}\mkern \@dotsep mu$}\hfill
        \hbox to\@pnumwidth{\@tocpagenum{#5}}\par}\fi}

\def\@tocpagenum#1{\devanagaripagenumber{#1}}
\makeatother

% Redefine the formatting of section entries in the table of contents
\makeatletter
\renewcommand{\l@section}[2]{%
  \ifnum \c@tocdepth >\z@
    \addpenalty\@secpenalty
    \addvspace{1.0em \@plus\p@}%
    \setlength\@tempdima{1.5em}% Adjust this
    \begingroup
      \parindent \z@ \rightskip \@pnumwidth
      \parfillskip -\@pnumwidth
      \leavevmode \normalsize
      \advance\leftskip\@tempdima
      \hskip -\leftskip
      {\devanagarifont #1}\nobreak\hfil 
      \nobreak\hb@xt@\@pnumwidth{\hss\devanagaripagenumber{#2}}\par
    \endgroup
  \fi}
\makeatother
%%%%%%%%%%%%%%%%%% Table of Contents Devanāgarī %%%%%%%%%%%%%%

% Redefine the footnote numbering to Devanagari
\renewcommand{\thefootnote}{\devanagarinumeral{footnote}}
% Use Devanagari numerals for footnote markers
\makeatletter
\renewcommand{\@makefnmark}{\hbox{\textsuperscript{\devanagarinumeral{footnote}}}}
\makeatother

